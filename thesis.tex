%%%%%%%%%%%%%%%%%%%%%%%%%%%%%%%%%%%%%%%%%%%%%%%%%%%%%%%%%%%
% EPFL report package, main thesis file
% Goal: provide formatting for theses and project reports
% Author: Mathias Payer <mathias.payer@epfl.ch>
%
% This work may be distributed and/or modified under the
% conditions of the LaTeX Project Public License, either version 1.3
% of this license or (at your option) any later version.
% The latest version of this license is in
%   http://www.latex-project.org/lppl.txt
%
%%%%%%%%%%%%%%%%%%%%%%%%%%%%%%%%%%%%%%%%%%%%%%%%%%%%%%%%%%%
\documentclass[a4paper,11pt,oneside]{report}
% Options: MScThesis, BScThesis, MScProject, BScProject
\usepackage[BScThesis]{EPFLreport}
\usepackage{xspace}

\title{The Future of Work in the Age of AI:\\ Displacement or Risk-Shifting?}
\author{André Ramon Zarza Tapia}




\newcommand{\sysname}{FooSystem\xspace}

\begin{document}
\maketitle



%%%%%%%%%%%%%%%%%%%%%%
\chapter{Analysis}
%%%%%%%%%%%%%%%%%%%%%%

The rhetoric of an AI driven displacement of people from their jobs is one that is all too broad for the AI uses of the world today. In this chapter, a more nuanced view of the situation is presented by discussing how AI in industry may affect workers through modes other than displacement. Firstly though, the author presents the task that AI is able to complete and why some menial ones can be considered far more complicated. Then the chapter introduces and explores 4 crucial alternatives to displacement in which firm are using AI to re-allocate risk from themselves onto their workers. Finally some potential policies to respond to both displacement and risk-shifting concerns are briefly discussed.

The main hypotheses in this chapter is that the view that tasks that are performed by workers are often routine, easy to follow and something that one can describe to a computer, yet that seems not to be the case. As explained in the text there exists plenty of tasks which are non-routine and hence harder or impossible to be mimicked by a robot or AI driven machine. This is because non-routine tasks were "deemed too difficult to program and dependent on skills like perception, problem-solving, and intuition". This idea introduced by the author is one that was carefully and well placed to emphasize the erroneous perception that AI is a means of getting rid of workers in order of efficiency. One can believe that it is purposely done to justify the ideals that AI only recompositions labor and that the rhetoric that AI will take over is far fetched. The authors argument on the hypotheses is strong as one can relate to their everyday task of varying importance and to the fact that though most tasks can be routine they require situational awareness and intuition.

Following the rhetoric, many forecasts predicted grim outcomes for employment with the implementation of AI systems in industry. Risk calculation like those of Frey Osborne estimated that 47\% of U.S jobs were at risk. However the reality is far more complicated. Further studies such as that done by McKinsey have demonstrated that automation "often leads not to the elimination of occupations, but tot changes in their task composition". Here, the author presents strongly arguments through use of solid examples the ambiguity of dominance that AI could have in industry. The idea that that AI will not displace jobs but rather restructure them is support throughout with the mechanisms they present that industries use to recomposition certain jobs.


Here, the idea presented is that AI will be implemented as a mean of reallocating risk from the employer to employee through different mechanism. What appears to be a mean of of rendering the work place more efficient can insidiously hide work by offloading its burden from firm onto its workers. The key idea behind algorithmic scheduling is that AI systems are capable of identifying efficiently the need of workers for a company at a given time. On of the supporting arguments as to why this mechanism reallocates burden is the increased pressure and stress imposed on workers subject to scheduling systems. However, there is a lack of supporting arguments where it could be beneficial. For instance in large systems such as that of medical wards the burden often falls on a single person and creates tough living standard for nurses and doctors alike. Hence, a scheduling system used in a beneficial manner to alleviate doctors and nurses from long strenuous shifts is not discussed giving it a biased outlook.

Moreover, firms us AI systems to define compensable work as demonstrated in the example of the sell of books on amazon where authors would be paid on a page-read basis. This is used a means of optimising compensation of workers by determining what is meaningful to the firm without taking into account the necessities to achieve said job. In the example of Uber, drivers have to replenish their cars with fuel as well as goodies such as  mints, water bottles, scented items and so on in order to achieve a good rating. All these are not deemed meaningful to the company hence they are not compensable actions. The burden shifted here is that of workers achieving tasks for which they are not compensated but are nevertheless indispensable as they help complete the required job. In this point the author effectively arguments that the imposition of uncompensated tasks related to the ability of a worker to do their job is strongly backed by examples. As seen for Uber or other companies where part of the job is getting there and workers are forced to count those as uncompensated hours lays a foundation for the introduction of a potential policy described in the final subsection of the chapter. Such policies could accurately value the entirety of the work it requires one to achieve the firms goals as an employee and be able to not fall victim to burdening the costs of carrying out the work they were hired for.

Finally, AI systems can be used as a tool to incentivise and evaluate worker productivity. In some cases productivity incentivisation is not a priori for workers who depend on commission. However in many contexts "fine-grained monitoring erodes trust, dignity and any sense of privacy from work, reduces workers' decisional autonomy and opens the door to labor exploitation by driving workers to the limits of their physical and mental capabilities". This point is one that could relate to the reader as the author explicitly express the potential dangers imposed on them by a supervision system that is intended to increase their productivity but instead achieves a feeling of privacy violations.

Furthermore, potential new policies are introduced and argued but the implementation of said policies is laking and rather cut short. This is why one can argue that the author, although effectively introducing and exploring their redirection of a doomsday point of view of machines taking over all our jobs to one of a simple job recomposition based on long standing work relations, lob sided. The lack there of and minimal exploration of the perspective that there was a positive implementation of said mechanisms makes the argumentation that there's a relocation of burden caused by AI incomplete. Nevertheless, the arguments were strong and well represented with the examples given hence, achieving a convincing ideology that AI driven systems will not entirely replace humans but rather reallocate burden by decomposition and restructuration of task.


\cleardoublepage
\phantomsection
\addcontentsline{toc}{chapter}{Bibliography}

% Appendices are optional
% \appendix
% %%%%%%%%%%%%%%%%%%%%%%%%%%%%%%%%%%%%%%
% \chapter{How to make a transmogrifier}
% %%%%%%%%%%%%%%%%%%%%%%%%%%%%%%%%%%%%%%
%
% In case you ever need an (optional) appendix.
%
% You need the following items:
% \begin{itemize}
% \item A box
% \item Crayons
% \item A self-aware 5-year old
% \end{itemize}

\end{document}