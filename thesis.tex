%%%%%%%%%%%%%%%%%%%%%%%%%%%%%%%%%%%%%%%%%%%%%%%%%%%%%%%%%%%
% EPFL report package, main thesis file
% Goal: provide formatting for theses and project reports
% Author: Mathias Payer <mathias.payer@epfl.ch>
%
% This work may be distributed and/or modified under the
% conditions of the LaTeX Project Public License, either version 1.3
% of this license or (at your option) any later version.
% The latest version of this license is in
%   http://www.latex-project.org/lppl.txt
%
%%%%%%%%%%%%%%%%%%%%%%%%%%%%%%%%%%%%%%%%%%%%%%%%%%%%%%%%%%%
\documentclass[a4paper,11pt,oneside]{report}
% Options: MScThesis, BScThesis, MScProject, BScProject
\usepackage[BScThesis]{EPFLreport}
\usepackage{xspace}

\title{The Future of Work in the Age of AI:\\ Displacement or Risk-Shifting?}
\author{André Ramon Zarza Tapia}




\newcommand{\sysname}{FooSystem\xspace}

\begin{document}
\maketitle



%%%%%%%%%%%%%%%%%%%%%%
\chapter{Analysis}
%%%%%%%%%%%%%%%%%%%%%%

The rethoric of an AI driven displacement of people from their jobs is one that is all too broad for the AI uses of the world today. In this chapter, a more nuanced view of the situation is presented by discussing how AI in industry may affect workers through modes other than displacement. Firstly though, the author presents the task that AI is able to complete and why some menial ones can be considered far more complicated. Then the chapter introduces and explores 4 crucial alternatives to displacement in which firm are using AI to re-allocate risk from themselves onto their workers. Finally some potential policies to respond to both displacement and risk-shifting concerns are briefly discussed.

Tasks that are performed by worked in the line of work can often be seen as routine, easy to retain and withouth the need for great skill or situational intuitiveness. This thought is what led engineering and exmployers to look for an alternative to workers to perform these tasks and be able to increase quantitiy of production hence increasing profits. However as explained in the text there exists plenty of tasks which are nonroutine and hence harder or impossible to be mimicked by a robot or AI driven machine. This is because nonroutine tasks were "deemed too difficult to program and denpendent on skills like perception, problem-solving, and intuition". This idea introduced by the author is one that was carefully and well placed to emphazise the erroneous perception that AI is a means of getting rid of workers in order of efficiency. Aprehensively one can believe that it is purposely done to justify the ideals that AI only recompositions labor and that the rehoric that AI will take over is far fetched.

Following the rethoric, many forecasts predicted grim outcomes for employement with the implementation of AI systems in industry. Risk calculation like those of Frey Osborne estimated that 47\% of U.S jobs were at risk. However the reality is far more complicated. Further studies such as that done by McKinsey have demonstrated that automation "oftens leads not to the elimination of occupations, but tot changes in their task composition". An example of this presented in the text is the well-known fear of ATM's replacing the need for Bank Tellers at all. Nevertheless, the itroduction of ATM's allowed for bank for cost-effective expansion of bank branches through the separation of tasks of bank tellers. Nevertheless in other industries such as retail, the boom of e-commerce thanks to ai-driven retail systems could create an imminent doomsday for approximately 75,000 stores by 2026. The impact of the implementation of these two instances of AI-drive systems demonstrates a way in which it is complicated to forecast to what extent AI will displace existing jobS. Regardless, it does demonstrate with certainty that condition of work will be impacted by AI.

Here the idea presented is that AI will be implemented as a mean of reallocating risk from the employer to employee through different mechanism. What appears to be a mean of of rendering the work place more efficient can insidiously hide work by offloading its burden from firm onto its workers. This is illustrated through 4 practices of which the first is algorithmic scheduling. The key idea behind algorithmic scheduling is that AI systems are capable of identifying efficiently the need of workers for a company at a given time. The example which illustrates this is the use of algorithmic scheduling in retail to allocate shifts to workers at surge times and ending their shift at points were a decreased number of workers is required. This pushes the burden of creating eficient schedules where works is maximized (i.e. workers are active in their roles at all times) to worker who must now face irregular and "split-shifting", high-fluctuation work schedules and "on-call shifts". This has a destabilizing effect on the worker's livelihoods by intefering with nonwork activites as weel as creating severe financial stress often leading to intergenerational cognitive harms. Fims meanwhile, face lower labor costs as a decrese in the risk of overstaffing and shifting the burden of demand to worker subjet to scheduling systems.

Moreover, firms us AI systems to define compensable work as demonstrated in the example of the sell of books on amazon where authors would be paid on a page-read basis. This is used a means of optimizing compensation of workers by determining what is meaningful to the firm without taking into account the necessities to achieve said job. In the example of Uber, drivers have to replenish their cars with fuel as well as goodies such as  mints, water bottles, well scented items and so on in order to achieve a good rating. All these are not deemed meaningful to the company hence they are not compensable actions. The burden shifted here is that of workers achieving tasks for which they are not compensated but are nevetherless indispensbable as they help complete the required job.

As a result of using AI systems to determine loss or fraud in a firm workers are monitored for theft and fraud and it is "often practicaly inseperable from traking productivity or efficiency purposes". The firms concerns of fraud can often be used to justify an entire data collection regime burdening the workers with constant monitoring. As well as monitoring systems can be used for hiring purposes and predictive tendencies when it comes to prediction of worker's capability of theft or fraud. These are often entangled with the employers personal biases. 

Finally, AI systems can be used as a tool to incentivize and evaluate worker productivity. In some cases producivity incentivization is not a priori for workers who depend on commision. However in many contexts "fine-grained monitoring erodes trust, dignity and any sense of privacy from work, reduces workers' decisional autonomy and opens the door to labor exploitation by driving workers to the limits of their physical and mental capabilities".

However, all these dynamics were not ones created by AI but rather long standing features of labor/management relations. AI is meerly a mean of enabling the pursuit of existing goals for firms and offload the burden of having to apply and implement these techniques to AI systems effectively reallocating risks from theselves onto workers.

In conclusion the chapter redirects the idea of the all too common rethoric that AI systems will displace many from their jobs to the idea that in fact the implementation of AI systems by firms will meerly create a reallocation of risk from employer to employee. The 4 presented techniques here effectively demonstrate and support this view by argumentating manners in which said mechanisms have changed and/or could and hence increase the burden carried by the worker. Nevertheless, it is mentioned that these problems of burden reallocation is one that was always part industries and loabor/management realtions. Therefore, the AI is not to blame for such shifts but that of social structure and insitutional structure as presented in the final paragraph by exploring the fault in certain laws and introducing potential policies.

\cleardoublepage
\phantomsection
\addcontentsline{toc}{chapter}{Bibliography}

% Appendices are optional
% \appendix
% %%%%%%%%%%%%%%%%%%%%%%%%%%%%%%%%%%%%%%
% \chapter{How to make a transmogrifier}
% %%%%%%%%%%%%%%%%%%%%%%%%%%%%%%%%%%%%%%
%
% In case you ever need an (optional) appendix.
%
% You need the following items:
% \begin{itemize}
% \item A box
% \item Crayons
% \item A self-aware 5-year old
% \end{itemize}

\end{document}